\documentclass[11pt]{article}
\usepackage{float}
%
%Margin - 1 inch on all sides
%
\usepackage[letterpaper]{geometry}
\usepackage{times}
\usepackage{amsmath}
\geometry{top=1.0in, bottom=1.0in, left=1.0in, right=1.0in}

\pagenumbering{gobble}

\usepackage{ragged2e}
\usepackage{mwe}

\usepackage{multicol}

\setlength{\RaggedRightParfillskip}{.25\textwidth plus 1fil}
\setlength{\RaggedRightRightskip}{0pt plus .1\textwidth}
\setlength{\RaggedRightParindent}{1em}


\usepackage{listings}
\usepackage{color}

\definecolor{dkgreen}{rgb}{0,0.6,0}
\definecolor{gray}{rgb}{0.5,0.5,0.5}
\definecolor{mauve}{rgb}{0.58,0,0.82}
\lstset{frame=tb,
    language=[x86masm]assembler,
    aboveskip=3mm,
    belowskip=3mm,
    showstringspaces=false,
    columns=flexible,
    basicstyle={\small\ttfamily},
    numbers=none,
    numberstyle=\tiny\color{gray},
    keywordstyle=\color{blue},
    commentstyle=\color{dkgreen},
    stringstyle=\color{mauve},
    breaklines=true,
    breakatwhitespace=true,
    tabsize=3
}


\begin{document}
    \begin{flushright}
        Benton Guess \\
        CSCE-451 \\
        Feb 12, 2020 \\
    \end{flushright}
\RaggedRight

\section*{Question 1}
Looking a the context of the function, we can see that it has at least 3 arguments.

\section*{Question 2}
These were all compiled with \texttt{gcc -m32 \dots} with gcc 9.1\\

\subsection*{Addition}

For all three of the addition operators, \texttt{a++}, \texttt{a = a + 1}, \texttt{a += 1}, they compile to the same instructions. However, amidst this we get a \texttt{nop} instruction which is unrelated to the question at hand, but this was interesting nonetheless.

C code
\lstset{frame=tb,
    language=C,
    aboveskip=3mm,
    belowskip=3mm,
    showstringspaces=false,
    columns=flexible,
    basicstyle={\small\ttfamily},
    numbers=none,
    numberstyle=\tiny\color{gray},
    keywordstyle=\color{blue},
    commentstyle=\color{dkgreen},
    stringstyle=\color{mauve},
    breaklines=true,
    breakatwhitespace=true,
    tabsize=3
}
\begin{lstlisting}
void func() {
    int a = 0;
    int b = 0;
    int c = 0;

    a++;
    b += 1;
    c = c + 1;
    
    return;
}
\end{lstlisting}

x86
\lstset{frame=tb,
    language=[x86masm]assembler,
    aboveskip=3mm,
    belowskip=3mm,
    showstringspaces=false,
    columns=flexible,
    basicstyle={\small\ttfamily},
    numbers=none,
    numberstyle=\tiny\color{gray},
    keywordstyle=\color{blue},
    commentstyle=\color{dkgreen},
    stringstyle=\color{mauve},
    breaklines=true,
    breakatwhitespace=true,
    tabsize=3
}
\begin{lstlisting}
    push ebp
    mov, ebp, esp
    sub ebp, 16
    mov dword_ptr [ebp - 4], 0
    mov dword_ptr [ebp - 8], 0
    mov dword_ptr [ebp - 12], 0
    add dword_ptr [ebp - 4], 1
    add dword_ptr [ebp - 8], 1
    add dword_ptr [ebp - 12], 1
    nop
    leave 
    ret
\end{lstlisting}
\subsection*{Conditionals}
Trying both the traditional \texttt{if \{...\} else \{...\}} statement and the tertiary operator, we get the same compiled code. Keeping the code as simple as possible, we can see that both statements take slightly different approaches to a conditional. The normal statement never makes use of a register, writing all of its instructions directly to the stack. The tertiary operator, on the other hand, stores its results in the eax register, then moves the eax register into the stack variable after the statement is done. The normal statement requires less operations, but it is hard to tell which produces more efficient code since they are so similar. \\
C code
\lstset{frame=tb,
    language=C,
    aboveskip=3mm,
    belowskip=3mm,
    showstringspaces=false,
    columns=flexible,
    basicstyle={\small\ttfamily},
    numbers=none,
    numberstyle=\tiny\color{gray},
    keywordstyle=\color{blue},
    commentstyle=\color{dkgreen},
    stringstyle=\color{mauve},
    breaklines=true,
    breakatwhitespace=true,
    tabsize=3
}
\begin{lstlisting}
int func(int num) {

    int a;
    int b;

    if (num == 1) {
        a = 10;
    } else {
        a = 100;
    }

    b = (num == 1) ? 10 : 100;

    return 0;
}

\end{lstlisting}

x86
\lstset{frame=tb,
    language=[x86masm]assembler,
    aboveskip=3mm,
    belowskip=3mm,
    showstringspaces=false,
    columns=flexible,
    basicstyle={\small\ttfamily},
    numbers=none,
    numberstyle=\tiny\color{gray},
    keywordstyle=\color{blue},
    commentstyle=\color{dkgreen},
    stringstyle=\color{mauve},
    breaklines=true,
    breakatwhitespace=true,
    tabsize=3
}
\begin{lstlisting}
    func:
    .LFB5:
            push    ebp
            mov     ebp, esp
            sub     esp, 16
            cmp     DWORD PTR [ebp+8], 1
            jne     .L2
            mov     DWORD PTR [ebp-4], 10
            jmp     .L3
    .L2:
            mov     DWORD PTR [ebp-4], 100
    .L3:
            cmp     DWORD PTR [ebp+8], 1
            jne     .L4
            mov     eax, 10
            jmp     .L5
    .L4:
            mov     eax, 100
    .L5:
            mov     DWORD PTR [ebp-8], eax
            mov     eax, 0
            leave
            ret
\end{lstlisting}
\subsection*{Loops}
Trying all three types of loops, the for, while, and do-while, gave significantly different results in terms of control flow. These were tested by creating a simple loop that increments a stack variable. First off, the do-while loop appears to be the most efficient in terms of operations and jumps. Since the condition check and the jump occur at the same point in the code, they can be chained into one operation. This means that at most the do-while loop will only perform at most \texttt{num - 1} jumps. The for loop seems to be the next most efficient, where we have both adds occurring in sequence then a quick comparison. However, since the code must start with a condition check, the for loop must do \texttt{num} jumps. The while loop plays out the strangest, with the start occuring like the for loop. However, we do not see the counter get incremented in the body. This makes since, since this while loop has the counter incremented in the condition. However, we see some compiler trickery as it uses \texttt{lea edx, [1 + eax]} to increment our counter. Since the counter is incremented after the comparison, we get a slightly more complicated conditional comparison, however the overall control flow is identical to the for loop. \\


C code
\lstset{frame=tb,
    language=C,
    aboveskip=3mm,
    belowskip=3mm,
    showstringspaces=false,
    columns=flexible,
    basicstyle={\small\ttfamily},
    numbers=none,
    numberstyle=\tiny\color{gray},
    keywordstyle=\color{blue},
    commentstyle=\color{dkgreen},
    stringstyle=\color{mauve},
    breaklines=true,
    breakatwhitespace=true,
    tabsize=3
}
\begin{lstlisting}
int func(int num) {
    int a = 0;
    int b = 0;
    int c = 0;

    for (int i = 0; i < num; i++) a++;

    int j = 0;
    while (j++ < num)
        b++;

    int k = 0;
    do {
        c++;
    } while (++k < num);

    return 0;
}
\end{lstlisting}

x86
\lstset{frame=tb,
    language=[x86masm]assembler,
    aboveskip=3mm,
    belowskip=3mm,
    showstringspaces=false,
    columns=flexible,
    basicstyle={\small\ttfamily},
    numbers=none,
    numberstyle=\tiny\color{gray},
    keywordstyle=\color{blue},
    commentstyle=\color{dkgreen},
    stringstyle=\color{mauve},
    breaklines=true,
    breakatwhitespace=true,
    tabsize=3
}
\begin{lstlisting}
func:
.LFB5:
        push    ebp
        mov     ebp, esp
        sub     esp, 32
        mov     DWORD PTR [ebp-4], 0
        mov     DWORD PTR [ebp-8], 0
        mov     DWORD PTR [ebp-12], 0
.LBB2:
        mov     DWORD PTR [ebp-16], 0
        jmp     .L2
.L3:
        add     DWORD PTR [ebp-4], 1
        add     DWORD PTR [ebp-16], 1
.L2:
        mov     eax, DWORD PTR [ebp-16]
        cmp     eax, DWORD PTR [ebp+8]
        jl      .L3
.LBE2:
        mov     DWORD PTR [ebp-20], 0
        jmp     .L4
.L5:
        add     DWORD PTR [ebp-8], 1
.L4:
        mov     eax, DWORD PTR [ebp-20]
        lea     edx, [eax+1]
        mov     DWORD PTR [ebp-20], edx
        cmp     DWORD PTR [ebp+8], eax
        jg      .L5
        mov     DWORD PTR [ebp-24], 0
.L6:
        add     DWORD PTR [ebp-12], 1
        add     DWORD PTR [ebp-24], 1
        mov     eax, DWORD PTR [ebp-24]
        cmp     eax, DWORD PTR [ebp+8]
        jl      .L6
        mov     eax, 0
        leave
        ret
\end{lstlisting}


\section*{Question 3}
The value returned should be equal to the function's first argument minus its second argument.

\section*{Question 4}
Function \texttt{\_Z3sixiii} has 3 arguments, and function \texttt{\_Z4fiveiii} has 3 arguments.

\section*{Question 5}
First, the "six" function calls the "five" function passing its 3 arguments to the function in the order of 3, 1, 2. The "five" functions 3 arguments, a, b, and c, are put through the following equation. In terms of "five", "six" arguments 1, 2, 3 are b, c, and a respectively.
\[
    n_1 = (a - c) + b
\]
\[
    n_2 = \begin{cases}
        0 & n_1 \geq 0\\
        -1 & n_1 < 0\\
    \end{cases}
\]

\[
    n_3 = \left\lfloor\frac{(n_1 * 1431655766)}{2^{32}}\right\rfloor - n_2
\]\text{else} 

$n_3$ is returned by the "five" function and the subsequently returned by the "six" function.

\section*{Question 6}
Note this is done on 8 bit signed/unsigned integers.
\begin{table}[H]
    \centering
    \begin{tabular}{|c|c|c|c|c|c|c|c|}
        \hline
        % REMEMBER TO COMPLEMENT THE SECOND
        Arg 1 & Arg 2 & CF & PF & AF & ZF & SF & OF\\
        \hline
        %Zeros
        \texttt{ 0x10 } & \texttt{ 0x00 } & 0 & 0 & 0 & 0 & 0 & 0 \\
        \texttt{ 0x80 } & \texttt{ 0x10 } & 0 & 0 & 0 & 0 & 0 & 1 \\
        \texttt{ 0x80 } & \texttt{ 0x00 } & 0 & 0 & 0 & 0 & 1 & 0 \\
        \texttt{ 0x10 } & \texttt{ 0x20 } & 0 & 0 & 1 & 0 & 0 & 0 \\
        \texttt{ 0x80 } & \texttt{ 0x10 } & 0 & 0 & 1 & 0 & 0 & 1 \\
        \texttt{ 0x90 } & \texttt{ 0x10 } & 0 & 0 & 1 & 0 & 1 & 0 \\
        \texttt{ 0x30 } & \texttt{ 0x00 } & 0 & 1 & 0 & 0 & 0 & 0 \\
        \texttt{ 0x80 } & \texttt{ 0x20 } & 0 & 1 & 0 & 0 & 0 & 1 \\
        \texttt{ 0x81 } & \texttt{ 0x00 } & 0 & 1 & 0 & 0 & 1 & 0 \\
        \texttt{ 0x00 } & \texttt{ 0x00 } & 0 & 1 & 0 & 1 & 0 & 0 \\
        \texttt{ 0x10 } & \texttt{ 0x10 } & 0 & 1 & 1 & 0 & 0 & 0 \\
        \texttt{ 0x80 } & \texttt{ 0x20 } & 0 & 1 & 1 & 0 & 0 & 1 \\
        \texttt{ 0x90 } & \texttt{ 0x20 } & 0 & 1 & 1 & 0 & 1 & 0 \\
        \texttt{ 0x00 } & \texttt{ 0x90 } & 1 & 0 & 0 & 0 & 0 & 0 \\
        \texttt{ 0x00 } & \texttt{ 0x20 } & 1 & 0 & 0 & 0 & 1 & 0 \\
        \texttt{ 0x00 } & \texttt{ 0x80 } & 1 & 0 & 0 & 0 & 1 & 1 \\
        \texttt{ 0x00 } & \texttt{ 0x81 } & 1 & 0 & 1 & 0 & 0 & 0 \\
        \texttt{ 0x00 } & \texttt{ 0x20 } & 1 & 0 & 1 & 0 & 1 & 0 \\
        \texttt{ 0x10 } & \texttt{ 0x81 } & 1 & 0 & 1 & 0 & 1 & 1 \\
        \texttt{ 0x00 } & \texttt{ 0xa0 } & 1 & 1 & 0 & 0 & 0 & 0 \\
        \texttt{ 0x00 } & \texttt{ 0x10 } & 1 & 1 & 0 & 0 & 1 & 0 \\
        \texttt{ 0x10 } & \texttt{ 0x80 } & 1 & 1 & 0 & 0 & 1 & 1 \\
        \texttt{ 0x00 } & \texttt{ 0x82 } & 1 & 1 & 1 & 0 & 0 & 0 \\
        \texttt{ 0x00 } & \texttt{ 0x10 } & 1 & 1 & 1 & 0 & 1 & 0 \\
        \texttt{ 0x10 } & \texttt{ 0x82 } & 1 & 1 & 1 & 0 & 1 & 1 \\
        
        \hline
    \end{tabular}
\end{table}

\section*{Question 7}
The code here is simply a static array being passed to a function, then summed based on the lower 16 bits of each array element.cThe final value of \texttt{EAX} will be 26862.

\section*{Question 8}
This function just does a bunch of operations on two parameters. The final value of \texttt{EAX}, the return value, will be 9000.

\section*{Fibonacci Code for Graduate Students}
Since I am in the honors section, I was not completely sure If I needed to do this one, so I did it just to be safe. The instructions to compile is as follows.
\begin{itemize}
    \item \texttt{nasm -f elf32 -g -F dwarf fib.s -o fib.o}
    \item \texttt{g++ -m32 -o fib fib.o -lc}
\end{itemize}

The binary runs as a CLA with an option of 0 or 1 arguments. Supplying no arguments to the binary will simply compute the 13th Fibonacci number. If a number is supplied as the first command line argument, it will calculate based on that degree. Note that the result will only be calculated with 0 as the 0th Fibonacci number and 1 as the 1st, and from there on the formula is as it is normally calculated.

\begin{lstlisting}
; nasm -f elf32 -g -F dwarf fib.s -o fib.o
; g++ -m32 -o fib fib.o -lc
global main

section .data
    result_text: db "The value of a %d degree fibonacci is %u",0x0a,0x0
    failed_text: db "Arguments: [fibonacci degeree (defaults to 13)]",0x0a,0x0

section .text
    extern printf
    extern atoi

; fib(degree : i32)
; returns the result into degree 
fib:
    push ebp
    mov ebp, esp
    sub esp, 16
    push dword [ebp + 8]
    pop dword [esp + 12] ; target
    mov dword [esp + 8], 0 ; n-2 value
    mov dword [esp + 4], 1 ; n-1 value
    mov dword [esp], 1 ; counter


    ; ebp is useless from this point until return
    ; deal with 0 and 1
    mov ebp, [esp+12]
    cmp ebp, 0 
    je fib_zero_cond
    cmp ebp, 1
    je fib_done

    ; deal with [2,47]
    fib_loop:
    mov ebp, [esp + 8]
    add ebp, [esp + 4]
    push ebp
    mov ebp, [esp + 8]
    mov [esp + 12], ebp
    pop ebp
    mov [esp + 4], ebp

    inc dword [esp]
    mov ebp, [esp]
    cmp ebp, [esp + 12]
    jl fib_loop
    jmp fib_done

    
    fib_zero_cond:
    mov dword [esp+4], 0

    fib_done:
    push dword [esp + 4]
    pop ebp 
    add esp, 16
    mov [esp + 8], ebp
    pop ebp
    ret

; Registers are used in main for operate functions like printf and atoi, as well as return the proper code
; However, only ebp and esp are used in fib
main:
    mov ebp, esp
    sub esp, 4

    ; Ensures that we have the right number of arguments
    cmp dword [ebp + 4], 2
    jg failed_code
    cmp dword [ebp + 4], 1
    jne has_argument
    mov dword [ebp - 4], 13
    jmp run_fib


    has_argument:
    ; Parse command line args using atoi
    ; get type of op
    mov edx, [ebp + 8]
    mov eax, [edx + 4]
    push eax
    call atoi
    add esp, 4
    mov [ebp - 4], eax

    run_fib:
    push dword [ebp - 4]
    call fib
    mov ebx, [esp]
    add esp, 4

    push ebx
    push dword [ebp - 4]
    push result_text
    call printf
    add esp, 12
    jmp Main_Done

    failed_code:
    push failed_text
    call printf
    add esp, 8
    mov eax, 1
    ret
    


    Main_Done:
    add esp, 4
    mov eax, 0
    ret
\end{lstlisting}

\end{document}